\documentclass{article}
\usepackage{bm}
\usepackage{amssymb}
\usepackage[thehwcnt=3]{iidef}
%\usepackage[left=2cm,right=2cm]{geometry}%for printing,broken for fancyhdr?
\usepackage{mathtools}
\usepackage{esdiff}
\DeclarePairedDelimiter\abs{\lvert}{\rvert}
\DeclarePairedDelimiter\norm{\lVert}{\rVert}
\DeclareMathOperator\sgn{sgn}
\def\emptyset{\varnothing}
\thecourseinstitute{HIT,Shenzhen}%Harbin Institute of Technology
\thecoursename{Differential Geometry and Topology}
\theterm{Spring 2018}
\begin{document}
\courseheader
\name{zhaofeng-shu33}
\begin{enumerate}
\item Show that applying an isometry of $\mathbb{R}^3$ does not change the first fundamental form. 
What is the effect of a dilation (i.e. a map $\mathbb{R}^3 \to \mathbb{R}^3$ given by $x\to ax$
for some constant a $\neq$ 0)?
\begin{proof}
An isometry in $\mathbb{R}^3$ has the form $f:x\to xP+a$. $(f\circ \sigma)_u =\sigma_u P,(f\circ \sigma)_v =\sigma_v P \Rightarrow 
E(f\circ\sigma) = \sigma_u PP^T\sigma_u^T =\sigma_u \sigma_u^T=E(\sigma)$. Similarly, $F,G$ are also unchanged under the isometry and the first fundamental form remains the same.

For dilation, $g: x\to ax$. $(g\circ \sigma)_u = a \sigma_u, (g\circ \sigma)_v = a \sigma_v \Rightarrow 
E(g\circ \sigma) = a^2 \sigma_u \cdot \sigma_u = a^2 E(\sigma)$.
Similarly, $F(g \circ \sigma) = a^2 F(\sigma), G(g\circ \sigma) = a^2 G(\sigma)$. If $ a = \pm 1$, the first fundamental form is unchanged; otherwise, it changes.
\end{proof}
\item Let $\gamma:(a,b) \to \mathbb{R}^3$ be a unit speed curve. The surface of tangent developable is given by
$\sigma(u,v) = \gamma(u)+v\gamma'(u)$
\begin{enumerate}[label=(\arabic*)]
\item Compute the first fundamental form of $\sigma$; Show that the first fundamental form is independent of the torsion of $\gamma$;
\item Show that the tangent developables of two curves $\gamma_1,\gamma_2$ are locally isometric if their curvature functions are the same;
\item Show that the tangent developable $\sigma$ is locally isometric to a plane.
\end{enumerate}
\begin{proof}\mbox{}
\begin{enumerate}[label=(\arabic*)]
\item $\sigma_u = \gamma'(u)+v\gamma''(u),\sigma_v = \gamma'(u)$. Since $\gamma'_u \circ \gamma''_u =0$, the first fundamental form is $(1+v^2\kappa^2)du^2+2dudv+dv^2$, where $\kappa$ is the curvature of the curve. From this expression, we see that the first fundamental form is indepedent with the torsion $\tau$ of $\gamma$. 
\item We further suppose both $\gamma_1, \gamma_2$ are regular. For $\sigma_1(u,v)$ on the tangent developable of $\gamma_1$, we map it to $\sigma_2(u,v)$ on the tangent developable of $\gamma_2$. 
Since $\gamma_1'\neq 0$, we can find a neighborhood $N_1 = N(\sigma_1(u,v),\epsilon_1)\cap \sigma_1$, such that
$\sigma_1^{-1}(N_1)$ and $N_1$ is one-to-one. 
Similarly we can find $\sigma_2(u,v) \in N_2 \subseteq \sigma_2$ such that $\sigma_2^{-1}(N_2)$ and $N_2$ is one-to-one.
Let $ K = \sigma_1^{-1}(N_1) \cap \sigma_2^{-1}(N_2)$, then $\sigma_1(K)$ and $\sigma_2(K)$ are one-to-one.
Therefore, we construct a locally smooth mapping $f$ from $\sigma_1(K)$ to $\sigma_2(K)$ as  $\sigma_2 \circ \sigma_1^{-1}$. 
The first fundamental form of $\sigma_1$ 
and $f\circ \sigma_1$ are the same from (1). It follows that $f$ is isometric.
\item By fundamental theorem of curves, it is possible to construct a planar curve with $\kappa(u)$ as curvature. From (2) $\sigma$ is locally
isometric to a plane. 
\end{enumerate}
\end{proof}
\item Show that Enneper's surface 
\begin{equation}
\sigma(u,v)=(u-{u^3\over 3}+uv^2, v-{v^3 \over 3}+vu^2,u^2-v^2)
\end{equation}
is conformally parametrized.
\begin{proof}
$\sigma_u = (1- u^2+v^2,2uv,2u), \sigma_v = (2uv,1-v^2+u^2,-2v)$.
The first fundamental form is $(1+u^2+v^2)^2 (du^2+dv^2)$, which is proportional to the first fundamental form of plane. Therefore, the surface is conformally parametrized.
\end{proof}
\end{enumerate}
\end{document}

