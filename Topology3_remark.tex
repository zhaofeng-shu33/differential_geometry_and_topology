\documentclass{article}
\usepackage{amsmath,amsthm,amssymb}
\usepackage{hyperref}
\usepackage{mathtools}
\DeclarePairedDelimiter\norm{\lVert}{\rVert}
\DeclarePairedDelimiter\abs{\lvert}{\rvert}
\DeclarePairedDelimiter\inner{\langle}{\rangle}
\def\emptyset{\varnothing}
\def\QEDclosed{\mbox{\rule[0pt]{1.3ex}{1.3ex}}}%hacked from IEEEtran.cls
\def\proofname{\normalfont \bfseries Proof}
\renewcommand\qedsymbol{\QEDclosed}
% resolve formula numbering
% all styles are plain, which is not optimal
% the official amsthm doc said we should use <definition> style for Definition,Example and Exercise.
\newtheorem{example}{Example}[section] % parent counter setting
\newtheorem{definition}{Definition}[section]
\newtheorem{exercise}{Exercise}[section]
\newtheorem{lemma}{Lemma}[section]
\newtheorem{remark}{Remark}[section]
\newtheorem{theorem}{Theorem}[section]
\usepackage{xpatch}%remove the dot after theorem env
\makeatletter
\AtBeginDocument{\xpatchcmd{\@thm}{\thm@headpunct{.}}{\thm@headpunct{}}{}{}}
\makeatother
\usepackage{enumitem}
\begin{document}
% 
\title{Lecture 3 remark: Compactness and Connectedness}
\author{zhaofeng-shu33}
\maketitle
\section{Compactness: definitions and examples}

\setcounter{theorem}{5}
\begin{theorem}
\begin{proof}\mbox{}

$\Rightarrow:$ Since a metric space is $C_1$.
\end{proof}
\end{theorem}
%\setcounter{definition}{3}
%\begin{definition}[Continued]
%We revise the condition as $(X,d)$ is a metric space and has the property that every sequence has a convergent subsequence. We show that $L>0$ and has the property that $B(x,r)\subseteq U_{i(x)},\forall x,\forall r<L$
%\begin{proof}
%We already know that $\varphi(x)>0,\forall x$. By \textbf{Lemma 5} the minimum on $X$ is achievable by the continuous function $\varphi$. Therefore $L=\varphi(x_0)>0$. Since $r<\phi(x),\forall x\Rightarrow \exists i(x),\,s.t. r<d(x,U^c_{i(x)}) \Rightarrow B(x,r)\subseteq U_{i(x)}$. Therefore $L$ is the supremum of such $r$. 
%\end{proof}
%\end{definition}
\section{Compactness: properties}
\setcounter{theorem}{9}
\begin{theorem}[Continued]
We show that a compact Hausdorff space $X$ is $T_4$.
\begin{proof}
Let $A,B$ be two closed set of $X$,
$\forall x\in A$, since $X$ is $T_3$, we can find $V_x \ni x, V_y\supseteq B$ and $V_x\cap V_y=\emptyset$. $\displaystyle\bigcup_{x\in A}V_x$ is an open cover of $A$, since $X$ is compact, we can find a finite subcover $\bigcup_{i=1}^n V_{x_i}$ covering $A$, which is disjoint with open set $\bigcap_{i=1}^n V_{y_i}$ covering $B$. Therefore $X$ is $T_4$.
\end{proof}
\end{theorem}
\section{Connectness}
\setcounter{lemma}{18}
\begin{lemma}\label{closedC}
If $B$ is a connected subset of $X$, then $\bar{B}$ is also connected.
\begin{proof}
Assume $\bar{B} = B_1 \dot\bigcup B_2,B_1$ and $B_2$ are open. $B = (B\cap B_1)\dot\bigcup (B\cap B_2)$, since $B$ is connected, either $B\cap B_1$ or
$B\cap B_2$ is empty. Suppose $B\cap B_1$ is empty, since $B\subseteq B_1\cup B_2$, $B\subseteq B_2$. $\exists x\in B_1\subseteq \bar{B}$. $x$ is an       
accumulated point of $B$. But $B_1$ is an open set containing $x$ and is disjoint with $B$, a contradiction.

\end{proof}
\end{lemma}
\setcounter{lemma}{21}
\begin{lemma}
Connected component of $X$ is closed.
\begin{proof}
Since $A$ is a connected component of $X$, by \textbf{Lemma \ref{closedC}} $\bar{A}$ is connected. $\bar{A}\subseteq A$, and by the definition of connected component, $\bar{A}=A$. Therefore $A$ is closed.
\end{proof}
\end{lemma}
\setcounter{example}{22}
\begin{example}[Continued]
If $X$ has only finitely many connecte components, then each component $A$ is both closed and open.
\begin{proof}
Let $X=\displaystyle\bigcup_{i=1}^n A_i$, then $A_i=(X\backslash A_i)^c = \displaystyle\bigcap_{\substack{j=1\\j\neq i}}^{n} A_j^c $
\end{proof}
\end{example}
\setcounter{lemma}{24}
\begin{lemma}
If $f: X \to Y$ is continuous and $X$ is path-connected, then $f(X)$ is also path-connected.
\begin{proof}
For any two points $x,y\in f(X)$, $f^{-1}(x),f^{-1}(y)\in X$(choose one if having multiple preimage). Then there exists a continuous curve $g:[0,1]\to X,g(0)=f^{-1}(x),g(1)=f^{-1}(y)$. $f\circ g$ is a continuous mapping from $[0,1]$ to $Y$, with $(f\circ g)(0)= x, (f\circ g)(1)=y$. Therefore $f(X)$ is path-connected.
\end{proof}
\end{lemma}
\begin{lemma}[Another Proof]
\begin{proof}
Assume $X=U_1 \dot\bigcup U_2,U_1$ and $U_2$ are open. Since $U_1,U_2$ are not empty, we can find $x\in U_1,y \in U_2$. 
Since $X$ is path-connected, we can find a continuous mapping from $[0,1]$ to $X$ such that $f(0)=x,f(1)=y$. Then
$[0,1]=f^{-1}(U_1 \cap f([0,1])) \dot\bigcup f^{-1}(U_2 \cap f([0,1])) $, a contradiction.
\end{proof}
\end{lemma}
\end{document}
