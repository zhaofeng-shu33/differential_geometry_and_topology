\documentclass{article}
\usepackage{bm}
\usepackage{amssymb}
\usepackage[thehwcnt=4]{iidef}
%\usepackage[left=2cm,right=2cm]{geometry}%for printing,broken for fancyhdr?
\usepackage{mathtools}
\usepackage{esdiff}
\DeclarePairedDelimiter\abs{\lvert}{\rvert}
\DeclarePairedDelimiter\norm{\lVert}{\rVert}
\DeclareMathOperator\sgn{sgn}
\def\emptyset{\varnothing}
\def\v#1{\overrightarrow{#1}}
\thecourseinstitute{HIT,Shenzhen}%Harbin Institute of Technology
\thecoursename{Differential Geometry and Topology}
\theterm{Spring 2018}
\begin{document}
\courseheader
\name{zhaofeng-shu33}
\begin{enumerate}
\item Prove that $S=\{(-\infty,a) | a$ is rational$\}$ is a topology basis of the real line $\mathbb{R}$ (with some appropriately defined topology).
\begin{proof}\mbox{}
\begin{enumerate}
\item $\displaystyle\mathbb{R}=\bigcup_{a\in Q} (-\infty,a)$
\item $(-\infty,a)\cap (-\infty,b)=(-\infty,\min\{a,b\})$
\end{enumerate}
The topology generated by $S$ is $T=\{(-\infty,b)|b\in \mathbb{R}\}\cup\{\emptyset,\mathbb{R}\}$
\end{proof}
\item Let $X=\mathbb{R}$ be the real line and $S$ the set of all irrational numbers. Define $T=\{U\backslash A | U \textrm{ is open in }\mathbb{R} \textrm{ and } A\subseteq S\}$.
\begin{enumerate}[label=(\alph*)]
\item Show that $T$ is a topology.
\item Show that $(X,T)$ has $T_2$, but not $T_3$ property.
\item Show that $(X,T)$ is first countable.
\item Prove that $S$ is a discrete subspace of $(X,T)$. Therefore, $S$ is not separable.
\item Prove that $(X,T)$ is not $C_2$.
\end{enumerate}
\begin{proof}\mbox{}
\begin{enumerate}[label=(\alph*)]
\item 
\begin{enumerate}[label=(\roman*)]
\item $X,\emptyset \in T$

\item $\displaystyle\bigcup_{i\in I} (U_i \backslash A_i) = (\bigcup_{i\in I}U_i)\backslash B,B\subseteq \bigcup_{i\in I}A_i\subseteq S $. Therefore, $\displaystyle\bigcup_{i\in I} (U_i \backslash A_i)\in T$
\item $(U_1\backslash A_1)\cap (U_2\backslash A_2) = (U_1\cap U_2)\backslash (A_1\cup A_2) \in T$
\end{enumerate}
\item $\forall x,y \in \mathbb{R},x\neq y$. Let $d={\abs{x-y} \over 2}$, then $x\in (x-d,x+d)\in T,y\in (y-d,y+d)\in T$ and $(x-d,x+d)\cap (y-d,y+d)=\emptyset$. Therefore, the topology satisfies $T_2$. 

Consider a rational point $p$ and a closed set $S$. Suppose $U\backslash A$ is an open neighborhood containing $p$, then all rational points within $U$ are in this set. Take an irrational number $q$ from $U$. For any open neighborhood $Y\backslash B$ containing $S$, $q\in Y$, where $Y$ is open in $\mathbb{R}$.  We can find a rational number within $Y$, sufficiently close to $q$ such that this rational number is also in $U$. Therefore $(U\backslash A)\cap (V\backslash B) \neq \emptyset$ and the topology defined in this problem is not $T_3$.
\item Let $\mathcal{N} = \{ \{x\}\cup B(x,{1\over n})\backslash S,n=1,2,\dots\}$. For a neighborhood $ U\backslash A $ of $ x $, we can find sufficiently large $n$ such that $B(x,{1\over n})\subseteq U \Rightarrow \{x\} \cup B(x,{1\over n})\backslash Q^c \subseteq U\backslash A $. Therefore, $(X,T)$ is $C_1$.
\item $\forall S_1\subset S,S_1 = S\cap (\mathbb{R}\backslash(S\backslash S_1))$ is open  in $S$. Thus $S$ has discrete topology. Each single point set is open. Therefore the accumulated point set of $A$ is empty. $\bar{A}=S\Rightarrow A=S$ and $A$ is uncountable. Hence $S$ cannot be separable. 
\item Assume $(X,T)$ is $C_2$, then $(S,T)$ is $C_2$. $C_2$ space is separable, a contradiction. 
\end{enumerate}
\end{proof}
\item Show that a compact metric space is separable and thus is $C_2$.
\begin{proof}
$\forall n\in \mathbb{N},X = \displaystyle\bigcup_{x\in X} B(x,{1\over n}) = \displaystyle\bigcup_{i=1}^{m(n)} B(x_{n_i},{1 \over n})$. We choose $A=\{x_{n_i}| n\in \mathbb{N},i=1,2,\dots,m(n)\}$. Then $A$ is countable, and we verify $\bar{A}=X$. $\forall x\in X\backslash A$ and a neighborhood $U$ of $x$, we can find sufficiently large $n$ such that $B(x,{1\over n})\subseteq U$. For this $n$, there exists $n_i$ such that 
$x \in B(x_{n_i},{1\over n})\Rightarrow x_{n_i}\in B(x,{1\over n})\Rightarrow U\cap A\neq \emptyset$. Therefore, $A$ is dense in $X$. And by known conclusion, seperable metric space is $C_2$.
\end{proof}
\item Let $\sigma: U\to \Sigma $ be a parameterization of a regular surface $\Sigma$ with an open subset $U\subseteq \mathbb{R}^2$.
\begin{enumerate}[label=(\arabic*)]
\item Prove that a surface $\Sigma$ is part of a plane if and only if its second fundamental form is always zero.
\item Prove that a surface $\Sigma$ is part of a sphere if and only if its second fundamental form $\Pi_p$ is a non-zero-constant multiple of its first fundamental form $I_p$, i.e. $\Pi_p = c I_p$ for some real number $c$ and each $p\in \Sigma$.
\end{enumerate}
\begin{proof}\mbox{}
\begin{enumerate}[label=(\arabic*)]
\item $\Rightarrow:$ the second partial derivative of $\sigma$ with respect to $u,v$ is zero, therefore the second fundamental form is zero.

$\Leftarrow: $ We know that $\sigma_u \cdot \v{n} = 0 \Rightarrow \sigma_{uu} \cdot \v{n} + \sigma_{u} \cdot \v{n}_u =0$. Since $L=\sigma_{uu}\cdot \v{n}=0$, we have $ \sigma_{u} \cdot \v{n}_u =0$. 
Also $\sigma_{uv} \cdot \v{n} + \sigma_{u} \cdot \v{n}_v =0 \Rightarrow \sigma_v \cdot \v{n}_u = 0$. From $\v{n}\cdot \v{n} = 1 \Rightarrow \v{n}\cdot \v{n}_u = 0 $. While $\v{n},\sigma_u,\sigma_v$ is an orthogonal basis of $\mathbb{R}^3$, $\v{n}_u=0$. By the same deduction, $\v{n}_v=0$. Then $\v{n}$ is a constant unit vector and integrate $\sigma_u\cdot \v{n}=0$ gives $\sigma \cdot \v{n}=c$, which is the equation of a plane in $\mathbb{R}^3$.
\item $\Rightarrow:$ See the next problem solution for detail.

$\Leftarrow:$ We know that $\binom{\v{n}_u}{\v{n}_v} = -BA^{-1} \binom{\sigma_u}{\sigma_v}$. Since $B=c A$, $\v{n}_u=-c \sigma_u, \v{n}_v = -c \sigma_v \Rightarrow \v{n} + c\sigma = a \Rightarrow \norm{\sigma-{a \over c}} = \abs{{1\over c}}$. That is, $\Sigma$ is part of a sphere.
\end{enumerate}
\end{proof}
\item Compute the Gauss curvature of the sphere $S^2$ of radius $r$.
\begin{proof}
We provide two methods:
\begin{enumerate}
\item Use the parameterization $(u,v)\longmapsto (r\cos u \cos v,r\cos u \sin v,r\sin u)$
The first fundamental form matrix $A = \begin{pmatrix} r^2 & 0 \\ 0 & r^2\cos^2 u \end{pmatrix}$.
The second fundamental form matrix $B = A$.
Therefore the Gaussian curvature $K=\det(BA^{-1})={1 \over r^2}$
\item The intersection curve $\gamma $ of a normal section with the sphere is always a great circle $\Rightarrow k_1 =k_2 =\pm {1\over r} \Rightarrow K= {1\over r^2}$. 
\end{enumerate}
\end{proof}
\end{enumerate}

\end{document}

