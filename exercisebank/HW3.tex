\begin{problem}
Show that applying an isometry of $\mathbb{R}^3$ does not change the first fundamental form. 
What is the effect of a dilation (i.e. a map $\mathbb{R}^3 \to \mathbb{R}^3$ given by $x\to ax$
for some constant a $\neq$ 0)?
\end{problem}
\begin{solution}
\begin{proof}
An isometry in $\mathbb{R}^3$ has the form $f:x\to xP+a$. $(f\circ \sigma)_u =\sigma_u P,(f\circ \sigma)_v =\sigma_v P \Rightarrow E(f\circ\sigma) = \sigma_u PP^T\sigma_v^T =\sigma_u \sigma_v^T=E(\sigma)$. Similarly, $F,G$ are also unchanged under the isometry and the first fundamental form remains the same.
\end{proof}
\end{solution}

\begin{problem}
Let $\gamma:(a,b) \to \mathbb{R}^3$ be a unit speed curve. The surface of tangent developable is given by
$\sigma(u,v) = \gamma(u)+v\gamma'(u)$
\begin{enumerate}[label=(\arabic*)]
\item Compute the first fundamental form of $\sigma$; Show that the first fundamental form is independent of the torsion of $\gamma$;
\item Show that the tangent developables of two curves $\gamma_1,\gamma_2$ are locally isometric if their curvature functions are the same;
\item Show that the tangent developable $\sigma$ is locally isometric to a plane.
\end{enumerate}
\end{problem}
\begin{solution}
\begin{proof}\mbox{}
\begin{enumerate}[label=(\arabic*)]
\item $\sigma_u = \gamma'(u)+v\gamma''(u),\sigma_v = \gamma'(u)$. Since $\gamma'_u \circ \gamma''_u =0$, the first fundamental form is $(1+v^2\kappa^2)du^2+2dudv+dv^2$, where $\kappa$ is the curvature of the curve. From this expression, we see that the first fundamental form is indepedent with the torsion $\tau$ of $\gamma$. 
\item 
\item We construct a planar curve with $\kappa(u)$ as curvature. By fundamental theorem of curves, it is possible. Then 
\end{enumerate}
\end{proof}
\end{solution}

\begin{problem}
Show that Enneper's surface 
\begin{equation}
\sigma(u,v)=(u-{u^3\over 3}+uv^2, v-{v^3 \over 3}+vu^2,u^2-v^2)
\end{equation}
is conformally parametrized.
\end{problem}
\begin{solution}
\begin{proof}
$\sigma_u = (1- u^2+v^2,2uv,2u), \sigma_v = (2uv,1-v^2+u^2,-2v)$.
The first fundamental form is $(1+u^2+v^2)^2 (du^2+dv^2)$, which is proportional to the first fundamental form of plane. Therefore, the surface is conformally parametrized.
\end{proof}
\end{solution}


